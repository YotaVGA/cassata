\mainlanguage[it]

\starttext

\section{Introduzione}
Questo documento descrive il funzionamento logico degli algoritmi di rendering
di Cassata.

I motivi per cui scrivo questo documento sono due. Il primo è quello di
attirare l'attenzione di persone (inizialmente sopratutto amici, ma verrà
tradotto anche in inglese per allargare il bacino di utenza) per ricevere
critiche, consigli e quant'altro nello sviluppo del software.

Il secondo motivo è quello di chiarirmi le idee, scrivendo in maniera sistemata
tutto.

Il documento è, e sarà per un po', confusionario e sparso, scritto di fretta e
senza molti riferimenti a letteratura od altro. Inoltre mancheranno moltissimi
particolari, in parte perché ancora non scritti, in parte perché ancora non
risolti. Inoltre è tremendamente suscettibile di errori e cambiamenti. Molte
dimostrazioni inoltre ancora non esistono o non sono state formalizzate, per il
momento mi affido al mio giudizio, in attesa di formalizzare tutto.

Cercherò comunque di dare le nozioni di base di quelle cose che non è
plausibile ritenere conosciute allo sviluppatore medio che ha una discreta
infarinatura di computer grafica, senza però voler essere esaustivo.

Se qualcuno avesse una qualunque idea, di qualunque tipo o volesse comunque
mettersi in contatto con me, e non sapesse come fare, può scrivermi
all'indirizzo \mbox{yota\_vga@users.sourceforge.net}.

\section{Cosa dev'essere in grado di fare Cassata}
Cassata è in grado di renderizzare, se sono presenti gli shader appositi,
praticamente tutto, entro limiti decisamente poco stringenti:

\startitemize
\item Per tutti i valori richiesti dall'algoritmo devono poter essere stimati
con un errore scelto sia in eccesso che in difetto (i valori richiesti
dipendono anche dagli shader usati, perciò non posso elencarli, ma normalmente
si tratta di emissività data una lunghezza d'onda, riflessività dato angolo
uscente, intensità luminosa entrante, angolo entrante e lunghezza d'onda, e
simili valori, normalmente richiesti in qualunque renderer).

\item La scena deve prevedere un solo punto di equilibrio, e questo dev'essere
stabile e statico.

\item Ogni quantità scelta dev'essere integrabile quasi ovunque (dove per quasi
ovunque si intende un insieme di misura nulla secondo Lebesgue) in un intorno
piccolo a piacere.

\item La scena può essere ad energia finita, od a potenza finita se vengono
date ulteriori informazioni (che saranno chiare più avanti nel documento).
\stopitemize

Tutto il resto si intende come possibile. Quindi praticamente ogni cosa può
essere renderizzata correttamente. Per sapere quanto può essere renderizzata
correttamente, bisogna parlare di rumore.

\subsection{Errori in Cassata}
Cassata incorpora un algoritmo adattativo che permette di migliorare
progressivamente la soluzione bene quanto si vuole. L'idea è quella di ridurre
l'errore in maniera tale che il valore cercato cada interamente all'interno di
uno degli intervalli di quantizzazione utilizzati nel risultato finale. Questo
non è possibile solo nel caso in cui l'elemento da trovare si trovi esattamente
nel valore che separa 2 intervalli di quantizzazione, e per questo è necessario
impostare un raggio che indichi qual'è il minimo errore tollerato.

Facendo in questo modo, quando l'elemento si trova a cavallo di 2 intervalli di
quantizzazione contigui, e l'errore è più piccolo del raggio scelto, la
procedura può terminare, indicando agli shader che si tratta di un possibile
valore intermedio.

È sempre possibile, chiaramente, trovare un raggio che renderizzi la scena alla
perfezione, basti considerare che gli elementi da quantizzare sono un numero
finito, ed ognuno si discosta di un valore dal punto di confine più vicino.
Escludendo gli elementi con distanza 0, che sono quelli che si trovano
esattamente a cavallo, basta prendere un raggio inferiore al più piccolo degli
altri per renderizzare tutto perfettamente. Tuttavia il raggio corretto non è
possibile trovarlo (per quanto ho potuto vedere fino ad ora, e sono abbastanza
convinto della sua impossibilità). Comunque scegliendo un raggio estremamente
piccolo si può ragionevolmente pensare che il rendering sia corretto.

Quindi al di la della scelta del raggio (che da luogo ad un errore di veramente
piccola entità, se il raggio è scelto come molto più piccolo degl'intervalli di
quantizzazione, e che inoltre può essere totalmente eliminato se viene scelto
un raggio sufficientemente piccolo) il rendering in Cassata risulterebbe
corretto al 100\%.

\stoptext
